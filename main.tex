\documentclass[a4paper]{article}
\title{Project Architectural \\ Scoping Doc}
\author{Keiran Harcombe}
\date{October 2022}
\begin{document}
	\maketitle
	
\section{Project Background} % Describe how this project came about, who is involved and the purpose

\section{Project Scope} % Project scope defines the boundaries of a project. Think of the scope as an imaginary box that will enclose all the project elements/activities. It not only defines what you are doing (what goes into the box), but it sets limits for what will not be done as part of the project (what doesn’t fit in the box). Scope answers questions including what will be done, what won’t be done, and what the result will look like. 

\section{High-Level Requirements} % Describe the high level requirements for the project

\section{Deliverables} % List agencies, stakeholders or divisions which will be impacted by this project and describe how they will be affected by the project

\section{Affected Parties} % List business processes or systems which will be impacted by this project and describe how they will be affected.

\section{Affected Business Processes or Systems} % Describe any specific components that are excluded from this project.

\section{Specific Exclusions from Scope} % Describe how you plan to implement the project. For example, will all parts of the project be rolled out at once or will it be incremental? What will be included in each release?

\section{Implementation Plan} % Include recommendations that lead to your proposed solution. Summarize what you’re proposing to do and how you’re going to meet the goals. You’ll be able to expand on the details within the ‘Our Proposal’ section.

\section{High-Level Timeline \& Schedule} % List business processes or systems which will be impacted by this project and describe how they will be affected.

\end{document}